% !TEX root = ../main.tex

\chapter{总结和展望}
我们正处于大数据时代,各种应用形态正不断地、越来越快地产生大量相互关联的数据。
图是一种用于建模数据实体之间关系的抽象数据结构,可以被抽象成图结构的数据称为图状结构数据。

图的更新、查询和分析是三类最为重要的图操作,图数据库通常实现了图的更新和查询,而图分析通常由离线的图分析系统完成。离线图分析系统通常具有较高的图分析性能,但它不可避免地牺牲了时效性,在线图分析系统同时实现了图的更新和分析,能够在图更新完成后的很短时间内对其进行分析,大大提高了图分析的时效性,但这要以大幅牺牲图分析性能为代价。

随着时间的推进,图数据可能是不断变化的。随着时间而变化的图叫做时序图。
普通图查询和图分析面向的通常是静态图或时序图的最新版本,而时序图查询不仅可以面向时序图的特定版本(包括最新版本和历史版本)进行基于图拓扑的查询,也可以基于时序图中顶点/边的生命周期信息进行查询。时序图分析同样既可以是在时序图的特定版本上的普通图分析任务(第一类时序图分析),也可以使用时序图的生命周期信息运行更为复杂的图分析算法(第二类时序图分析)。
属性图、RDF图和超图是定义图结构的三大模型范式,它们都有各自的时序版本。面向时序RDF图和时序超图的时序图查询并不是热点研究方向,相关系统研究工作不多,且性能难以令人满意。

本文基于上述背景,针对现有时序图处理系统的不足,设计并实现了一个高效的时序图处理系统\sys。

\section{全文总结}
本文首先介绍了时序图处理系统的研究现状和相关的技术背景,然后详细介绍了\sys 的设计和实现,最后通过一系列实验对系统的性能进行全面的评测。

\sys 是一个同时支持时序图查询和第一类时序图分析的图处理系统,它由时序图查询和时序图分析两大模块组成。
\sys 将时序属性图、时序RDF图和时序超图三种时序图模型集成到同一系统中。
具体来说,\sys 的时序图查询模块同时支持时序RDF图和时序超图两种时序图模型,而时序图分析模块则是基于时序属性图模型实现的。

\sys 系统运行在一个由RDMA网络相互连接的集群环境中,集群包含查询节点、分析节点和字符串服务器节点三类节点。
查询节点用于执行来自用户的时序RDF图查询请求和时序超图查询请求,分析节点用于处理图事务和时序图分析请求,字符串服务器节点用于处理字符串和整型ID之间的转换。
\sys 的时序图查询模块采用了去中心化的分布式架构,而时序图分析模块则是单机的。
每个查询节点和分析节点都由存储层、引擎层和接口层三层组成。存储层负责图数据的存储,引擎层负责图更新、查询和分析任务的执行,接口层负责接收来自于用户或系统管理员的请求,将它们解析为系统的内部表示,然后分配给特定节点的引擎层执行。接口层由分析节点和查询节点上的若干代理线程和为每个代理线程准备的专用缓冲区来实现,代理线程和工作线程之间可以通过单边RDMA操作实现高效的线程间通信,从而实现请求的分配和结果的获取。

时序图查询方面,本文在RDF图查询语言SPARQL的基础上设计了时序RDF图查询语言SPARQL-T,从零开始设计了时序超图查询语言HQL-T。时序 RDF 图存储和时序超图存储的设计思路类似,都使用了分布式键值存储辅以分布式排序数组的存储结构,键值存储都使用了直接索引和间接索引相结合的设计,在实现了高效的拓扑查询和时间条件查询的同时,减少了时序数据存储占用的空间。SPARQL-T的时序三元组时间范围模式和HQL-T的时序超边时间范围模式可以用来在特定条件下加速时间条件查询。SPARQL-T和HQL-T查询引擎都沿用了Wukong使用的图探索和全历史剪枝算法,都使用了fork-join机制来加速复杂查询的执行。

时序图分析方面,系统使用了一个高效可更新的、基于时序属性图模型的图存储结构\newstore,它使用段(一块连续的内存空间)来管理固定数量顶点的所有特定类型的邻边,这种设计能够显著提高扫边性能。为了减少时序数据带来的内存使用和额外计算开销,\newstore 使用了一种基于epoch的粗粒度MVCC机制。系统的时序图分析模块提供了只读事务RO\_TXN,时序图分析算法都是基于 RO\_TXN实现的。

本文从多个角度对\sys 的性能进行全面的评测,实验结果表明,系统能够实现时序RDF图查询和时序超图查询的高效处理,也能够在保证良好图事务处理性能的同时,实现第一类时序图分析的高效执行。

\section{未来工作展望}
大数据时代的背景下,时序图数据在产业界正得到越来越重要的应用,针对时序图数据的查询和分析是一个非常值得关注的研究领域。

\sys 虽然是一个高效的时序图处理系统,但它依然有其局限性。时序图查询方面,时序RDF图和时序超图存储都是从预先准备好的数据集中加载得到的,系统并不支持对时序RDF图和时序超图存储的更新,让时序图查询模块同样支持图的事务化更新,使其具备图数据库的功能是未来的一个工作方向。时序图分析方面,系统的时序图分析模块是单机的,要想支持更大规模图数据的存储和分析,系统就必须对其进行多机扩展,这同样是未来的一个工作方向。