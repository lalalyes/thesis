% !TEX root = ../main.tex

% \section{脚注}

% Lorem ipsum dolor sit amet, consectetur adipiscing elit, sed do eiusmod tempor
% incididunt ut labore et dolore magna aliqua. \footnote{Ut enim ad minim veniam,
% quis nostrud exercitation ullamco laboris nisi ut aliquip ex ea commodo
% consequat. Duis aute irure dolor in reprehenderit in voluptate velit esse cillum
% dolore eu fugiat nulla pariatur.}
\chapter{绪论}
\section{研究背景}
我们正处于大数据时代,物联网、社交网络、电子商务等应用形态正不断地、越来越快地产生大量数据。
大数据应用产生的数据实体之间通常是具有关联关系的,充分地利用这些数据,从中挖掘出尽量多有价值的信息是十分有意义的。

图(graph)是一种用于建模数据实体之间关系的抽象数据结构,它非常适用于表示有关联关系的数据。
通常可以使用$G=(V,E)$来表示一个图结构,其中$V$是图$G$中顶点的集合,$E$是图$G$中边的集合。
每个顶点都表示一个数据实体;每条边都将两个顶点连接起来,表示两个实体之间的关系。
图可以分为有向图和无向图两类。
有向图中的边是有方向的,从起始顶点指向目标顶点;无向图中的边是没有方向的。
除特别说明外,本文中提到的图都是有向图。可以被抽象成图结构的数据称为图状结构数据。
图论是数学的一个重要分支,研究图的性质等一切与图有关的问题。
计算机科学对图的研究以图论为理论基础,以提高图状结构数据的存储、更新、查询和分析等操作的效率为主要研究目标。

图查询和图分析是两类不同的图处理任务,它们的目标都是从图状结构数据中挖掘出特定信息,但它们要解决的问题区别很大。
图查询是根据查询请求指定的查询条件在图中寻找与之匹配的数据,不同查询的复杂性不尽相同,可以是获取特定顶点或特定边的相关信息等的简单查询,也可以是查询所有具有特定拓扑结构的子图等的复杂查询。
例如在社交网络中,如果用无向图来建模人与人之间的朋友关系,那么查询“所有朋友数量为5的人”就相对比较复杂,而查询“小明的朋友列表”则比较简单。
图分析任务通常是全图级别的计算,它基于某个特定的算法,对图中顶点或者边的属性以迭代的方式进行遍历、更新等操作,从而挖掘出图的某些全局或局部特征。
图分析任务的复杂性较高,每个任务需要迭代几轮到几十轮,甚至更多。
典型的图分析问题包括网页排序\cite{pagerank}(PageRank)、单源最短路径(SSSP)和强连通分量(SCC)等。 
Google的Pregel\cite{pregel}是图分析领域的开山之作,它采用了“以顶点为中心”的编程模型和一种称为BSP的计算模型,使得大规模图上的图分析能够简洁而高效地得以实现。

图的更新、查询和分析是三类最为重要的图操作,图数据库通常实现了图的更新和查询,而图分析由于其计算的复杂性,通常需要先从图数据库中导出图快照,然后加载到专用的图分析系统中进行离线图分析,这一过程称为ETL(抽取-转换-加载,Extract-Transform-Load)。离线图分析系统通常具有较高的图分析性能,但它不可避免地牺牲了时效性,图的更新需要经过非常耗时的ETL过程才能应用到图分析系统中,为了解决这一问题,在线图分析系统应运而生。在线图分析系统同时实现了图的更新和分析,能够在图更新完成后的很短时间内对其进行分析,无需耗时的ETL过程,大大提高了图分析的时效性。我们将各种图数据库、图查询和图分析系统统称为图处理系统,或称图计算系统。

随着时间的推进,图数据可能是不断变化的。随着时间而变化的图叫做时序图(temporal graph),或称动态图(dynamic graph)或进化图(evolving graph)。
图数据库和在线图分析系统管理的图就是典型的时序图,因为这些系统的图数据会随着图更新操作的执行而不断变化。带有生命周期信息的图状结构数据通常也可以抽象成时序图,这类数据在金融\cite{Haslhofer2016OBW}、路网\cite{COOKE1966493}和流行病学等领域正起着越来越重要的作用。
普通图查询和图分析面向的通常是静态图或时序图的最新版本,而时序图查询不仅可以面向时序图的特定版本(包括最新版本和历史版本)进行基于图拓扑的查询,也可以基于时序图中顶点/边的生命周期信息进行查询。
时序图分析同样既可以是在时序图的特定版本上的普通图分析任务(第一类时序图分析),也可以使用时序图的生命周期信息运行更为复杂的图分析算法(第二类时序图分析),例如时序单源最短路径(Temporal SSSP)、最晚出发时间(LD)\cite{type2tgp}等。
我们将支持时序图查询或时序图分析的系统统称为时序图处理系统。

\section{国内外研究现状}
\subsection{图模型和图处理系统}
图数据库和图分析系统长期以来都是学术界和工业界的研究热点,也涌现了一批代表性工作,例如图数据库Neo4j\cite{neo4j}、OrientDB\cite{OrientDB}和NebulaGraph\cite{nebula},离线图分析系统GraphLab\cite{graphlab}、GraphChi\cite{GraphChi}、PowerGraph\cite{PowerGraph}、GraphX\cite{GraphX}和Gemini\cite{Gemini}以及在线图分析系统LiveGraph\cite{LiveGraph}和GeaFlow\cite{GeaFlow}等。
也有一些只支持图查询、不支持图更新的图查询系统的研究工作,例如DEX\cite{dex}、Wukong\cite{wukong}等,这类系统虽然牺牲了图数据的可变性,但通常具有较好的图查询性能。

属性图\cite{Angles2018ThePG}、RDF\cite{rdf}图和超图是定义图结构的三大模型范式,不同的图处理系统往往会根据其要解决的问题选择其中一种作为其数据模型。

属性图模型相对比较简单,但表达能力很强,大部分图数据库和图查询系统、几乎所有图分析系统都是基于属性图模型实现的。

RDF相对比较复杂,但具有更强的表达能力和灵活性。RDF数据集的基本组成单元是三元组,每个三元组都由主语(subject)、谓词(predicate)和宾语(object)三个部分组成。
目前已经有不少面向RDF的图数据库和图查询系统,其中有些系统使用关系模型来存储三元组数据,例如TriAD\cite{triad}等,这种实现最大的问题在于查询的执行需要依赖三元组的连接这种开销比较大的操作来实现,
为了提高查询性能,这些系统通常需要使用提前剪枝、连接顺序选择和查询缓存等优化技术。
更好的方案是使用图模型来存储RDF图并使用图搜索算法来实现图查询,AllegroGraph\cite{allegro}、Trinity.RDF\cite{trinity}和 Ontotext GraphDB\cite{graphdb}等系统都采用了这种方案。
Wukong\cite{wukong}是一个先进的分布式RDF图查询系统,它同样采用了该方案,它还围绕RDMA(Remote Direct Memory Access,远端内存直接访问)这一高性能网络硬件技术,通过一系列存储和查询引擎上的优化技术,实现了RDF图的可扩展存储和低延迟、高并发的图查询请求处理。
与Trinity.RDF相比,Wukong处理单条查询的性能提高了10倍以上,查询吞吐量提高了上百倍。

超图(hypergraph)是在图的基础上泛化的一种数据结构,可以用来描述多元关系。
虽然超图理论在上世纪就已经发展完善,但面向超图的图处理系统的研究依然鲜有人涉足。

\subsection{时序图处理系统}
在关系型数据库中引入时间维度早已成为数据库领域的研究热点\cite{599934}\cite{tdb}\cite{itd}。数据库可能包含三类时间数据:用户自定义时间、有效时间和事务时间。
用户自定义时间没有特殊语义,由用户规定其含义,例如Jisoo的生日是1995年1月3日,那么1995-1-3就是用户自定义时间,系统可以把用户自定义时间和普通属性同等看待,无需特别处理。
有效时间指的是应用程序角度一个事件发生或一个事实成立的时间,可以是时间点、时间区间等,例如李华在2017到2021年期间是一名本科生,那么2017-2021年就是“李华是本科生”这一事实成立的有效时间。
事务时间是数据库对数据对象进行插入、删除或修改等操作的时间,它由数据库系统的内部时钟自动生成。
大多数主流关系型数据库,例如Oracle、MySQL和PostgreSQL等,仅支持对最新版本数据的查询和更新,历史版本的数据对外部是不可见的,且可能被系统的垃圾回收机制清理,这类数据库都是非时序数据库。
时序数据库既可以是支持有效时间的查询和更新的事实数据库(例如InfluxDB\cite{influx}等),也可以是支持使用事务时间查询历史版本数据的回滚数据库,亦可以是同时支持上述两类操作的双时序数据库(例如TimeDB\cite{timedb}等)。

时序关系型数据库中的时序概念同样适用于图处理系统。
与属性图、RDF图和超图相对应,时序属性图、时序RDF图和时序超图构成了定义时序图结构的三大范式。

在时序属性图中,顶点、边、顶点的各属性和边的各属性都可以有对应的生命周期,生命周期既可以是有效时间,也可以是事务时间。
在线图分析系统和大多数图数据库严格来说都是基于使用事务时间来表示生命周期的时序属性图模型实现的,但它们通常不能归类为时序图处理系统,因为它们面向的通常是图数据的最新稳定版本,不支持时序图查询或时序图分析。
TGraph\cite{TGraph}是少有的支持时序图查询的图数据库之一,它是基于Neo4j实现的。
支持第一类时序图分析的系统(例如Chronos\cite{Chronos}和SAMS\cite{sams}等)和支持第二类时序图分析的系统(例如Graphite\cite{Graphite}和Tink\cite{tink}等)通常都是基于使用有效时间来表示生命周期的时序属性图模型实现的。

时序RDF图模型并没有统一的标准,Gutierrez等人\cite{trdf}于2005年首次提出扩展RDF模型对有效时间的支持,并在2007年的另一篇文章\cite{trdfex}中做了进一步完善。他们提出将RDF三元组扩展为四元组$(s,p,o)[t]$的形式,其中$t$是一个时间戳。
Tappolet等人\cite{taosparql}设计了$\tau$-SPARQL语言来查询时序RDF图中的有效时间,并给出了从$\tau$-SPARQL到标准SPARQL的转换方法。
Bereta等人\cite{strdf}也提出了自己的时序RDF图模型stRDF和对应的查询语言stSPARQL。
时序RDF图的概念提出之后,也出现了一些面向时序RDF的图查询系统\cite{HyperBit}\cite{Yan2020TemporalRD},但这类系统数量不多,且在处理大规模(例如百万级以上数量三元组)的时序RDF图数据集上的查询时,查询响应时间通常达到毫秒级,甚至秒级,也难以有效地应对大量用户同时发起时序图查询请求的高并发场景。

时序超图方面,由于时序超图模型也没有统一的标准,加之基于超图模型的图处理系统方面的研究本身就很少,因此,针对时序超图的系统研究工作就更鲜有人涉足。

\section{本文工作}
针对时序图查询和时序图分析系统的研究现状,本文想要解决的问题包括:
\begin{enumerate}
    \item 面向时序RDF图和时序超图的时序图查询并不是热点研究方向,相关系统研究工作不多,且性能难以令人满意。
    \item 在线图分析系统虽然具备良好的时效性,但这要以大幅牺牲图分析性能为代价。以LiveGraph为例,它在实现图的事务化更新的同时,也保证了临接列表扫描这一图分析中最常用操作的良好的数据局部性,尽管如此,与最先进的离线图分析系统Gemini相比,它的图分析性能依然有40\%以上的下降。
\end{enumerate}

为了解决以上问题,本文设计并实现了一个分布式时序图处理系统\sys,它由时序图查询和时序图分析两大模块组成。时序图查询模块运行在多台机器上,负责时序图查询请求的处理;时序图分析模块运行在单台机器上,负责处理图的事务化更新和实时的第一类时序图分析任务。
时序图查询模块同时支持时序RDF图和时序超图数据集的存储和查询,它采用了分布式键值存储辅以分布式排序数组的存储结构,实现了时序RDF图和时序超图数据集的可扩展存储;它通过一系列查询引擎的优化,实现了时序图查询的低延迟、高并发处理。时序图分析模块使用时序属性图模型作为其数据模型,它通过一个高效的动态图存储结构\newstore 和一种基于epoch的粗粒度多版本并发控制机制,实现了在保证良好的图更新效率的同时,达到接近于离线时序图分析系统的图分析性能。

\section{论文结构}
本文共分为七章,本章是第一章,主要介绍了本文的研究背景以及与本文相关的研究工作,梳理出现有时序图查询和时序图分析系统亟待解决的问题,最后概括了本文的研究内容和结构安排。

第二章介绍了与本文相关的技术背景:首先对图的三大模型范式属性图、RDF图和超图进行了简要介绍,然后分别给出了时序属性图、时序RDF图和时序超图的定义和表示方法。由于系统的时序图查询模块是在Wukong的基础上实现的,所以该章从存储结构和查询引擎两个方面对Wukong进行了简要介绍。最后介绍了CSR和邻接列表两种图分析系统常用的图存储结构,并分析了它们各自的优劣势。

第三章首先介绍了\sys 的总体架构、运行环境和基本配置,然后简要概括系统中存储层、引擎层和接口层各自的作用,最后对系统的接口层进行了详细介绍。

第四章介绍了系统时序图查询模块的设计与实现。首先介绍了本文设计的时序RDF图查询语言SPARQL-T和时序超图查询语言HQL-T的语法和语义,然后先后介绍了时序RDF图的存储结构和查询引擎、时序超图的存储结构和查询引擎。

第五章介绍了系统时序图分析模块的设计与实现。首先详细介绍了一个高效可更新的时序属性图存储结构\store,然后介绍模块使用的一种粗粒度多版本并发控制机制和\store 结合该机制的修改后版本\newstore,最后介绍了时序图分析模块的只读事务,并简要说明如何基于只读事务在图分析引擎中实现第一类时序图分析。

第六章是\sys 的性能评测,通过一系列实验对系统时序图查询模块和时序图分析模块的性能进行全面的评测。

第七章对全文内容进行了总结,并展望了未来工作的方向。

